%! Author = yanni
%! Date = 17.10.2021

\documentclass[a4paper,12pt]{article}
\usepackage{fancyhdr}
\usepackage{fancyheadings}
\usepackage[ngerman,german]{babel}
\usepackage{german}
\usepackage[utf8]{inputenc}
%\usepackage[latin1]{inputenc}
\usepackage[active]{srcltx}
\usepackage{algorithm}
\usepackage[noend]{algorithmic}
\usepackage{amsmath}
\usepackage{amssymb}
\usepackage{amsthm}
\usepackage{bbm}
\usepackage{enumerate}
\usepackage{graphicx}
\usepackage{ifthen}
\usepackage{listings}
\usepackage{struktex}
\usepackage{hyperref}
\usepackage[T1]{fontenc}

%%%%%%%%%%%%%%%%%%%%%%%%%%%%%%%%%%%%%%%%%%%%%%%%%%%%%%
%%%%%%%%%%%%%% EDIT THIS PART %%%%%%%%%%%%%%%%%%%%%%%%
%%%%%%%%%%%%%%%%%%%%%%%%%%%%%%%%%%%%%%%%%%%%%%%%%%%%%%
\newcommand{\Fach}{Wahrscheinlichkeitstheorie für Inf.\ \& Lehramt}
\newcommand{\Name}{Yannick Brenning}
\newcommand{\Seminargruppe}{H}
\newcommand{\Matrikelnummer}{3732848}
\newcommand{\Semester}{WiSe 21/22}
\newcommand{\Uebungsblatt}{1} %  <-- UPDATE ME
%%%%%%%%%%%%%%%%%%%%%%%%%%%%%%%%%%%%%%%%%%%%%%%%%%%%%%
%%%%%%%%%%%%%%%%%%%%%%%%%%%%%%%%%%%%%%%%%%%%%%%%%%%%%%

\setlength{\parindent}{0em}
\topmargin -1.0cm
\oddsidemargin 0cm
\evensidemargin 0cm
\setlength{\textheight}{9.2in}
\setlength{\textwidth}{6.0in}

%%%%%%%%%%%%%%%
%% Aufgaben-COMMAND
\newcommand{\Aufgabe}[1]{
        {
        \vspace*{0.5cm}
        \textbf{Aufgabe #1}
        \vspace*{0.2cm}
    }
}
%%%%%%%%%%%%%%
\hypersetup{
    pdftitle = {\Fach{}: Übungsblatt \Uebungsblatt{}},
    pdfauthor = {\Name},
    pdfborder = {0 0 0}
}

\lstset{ %
    language=java,
    basicstyle=\footnotesize\tt,
    showtabs=false,
    tabsize=2,
    captionpos=b,
    breaklines=true,
    extendedchars=true,
    showstringspaces=false,
    flexiblecolumns=true,
}

\title{Übungsblatt \Uebungsblatt{}}
\author{\Name{}}

\begin{document}
    \thispagestyle{fancy}
    \lhead{\Fach{} \\ \small \Name{} - \Matrikelnummer{}}
    \rhead{\Semester{} \\  Übungsgruppe \Seminargruppe{}}
    \vspace*{0.2cm}
    \begin{center}
        \LARGE \textbf{Übungsblatt \Uebungsblatt{}}
    \end{center}
    \vspace*{0.2cm}

%%%%%%%%%%%%%%%%%%%%%%%%%%%%%%%%%%%%%%%%%%%%%%%%%%%%%%
%% Insert your solutions here %%%%%%%%%%%%%%%%%%%%%%%%
%%%%%%%%%%%%%%%%%%%%%%%%%%%%%%%%%%%%%%%%%%%%%%%%%%%%%%

    \Aufgabe{1}
    \begin{enumerate}[(a)]
        \item \((A \cup B \cup C)^{\mathrm{c}}\)
        \item \(A \cup B \cup C\)
        \item \((A \cap B^{\mathrm{c}} \cap C^{\mathrm{c}}) \cup
                (A^{\mathrm{c}} \cap B \cap C^{\mathrm{c}}) \cup
                (A^{\mathrm{c}} \cap B^{\mathrm{c}} \cap C^{\mathrm{c}})\)
        \item \((A \cap B) \cup (B \cap C) \cup (C \cap A)\)
        \item \((A \cap B \cap C^{\mathrm{c}}) \cup
                (B \cap C \cap A^{\mathrm{c}}) \cup
                (C \cap A \cap B^{\mathrm{c}})\)
        \item \((A \cap B \cap C^{\mathrm{c}}) \cup
        (B \cap C \cap A^{\mathrm{c}}) \cup
        (C \cap A \cap B^{\mathrm{c}})\)
    \end{enumerate}

    \Aufgabe{2}

    \Aufgabe{3}
    \begin{enumerate}[(a)]
        \item Es beschreibt, dass sowohl das erste als auch das zweite System störungsfrei arbeitet.
        \item
        \begin{enumerate}[A:]
            \item $ S_1 \cap S_2 \cap S_3 $
            \item $ (S_1 \cap S_2 \cap S_3)^{\mathrm{c}}$
            \item $ S_1 \cup S_2 \cup S_3 $
            \item $ (S_1 \cap S_2^{\mathrm{c}} \cap S_3^{\mathrm{c}})
                    \cup (S_1^{\mathrm{c}} \cap S_2 \cap S_3^{\mathrm{c}})
                    \cup (S_1^{\mathrm{c}} \cap S_2^{\mathrm{c}} \cap S_3)$
            \item $ (S_1^{\mathrm{c}} \cap S_2^{\mathrm{c}} \cap S_3^C)^{\mathrm{c}} $
        \end{enumerate}
        \item $ \Omega = {(0,0,0), (1,0,0), (0,1,0), (0,0,1)
        (1,1,0), (1,0,1), (0,1,1), (1,1,1)}$
        \item A und B sind Elementarereignisse.
        \item C besteht aus 7 Elementarereignissen, w\"ahrend D aus 3 besteht.
    \end{enumerate}

    \Aufgabe{4}\\
    \begin{itemize}
        \item
        $ P(B^{\mathrm{c}}) = 1 - P(B) = 1 - 0.45 = 0.55 $ \\
        $ P(A \cup B) = P(A) + P(B) - P(A \cap B) \Rightarrow P(A \cap B) = P(A) + P(B) - P(A \cup B) $ \\
        $ P(A \cap B^{\mathrm{c}}) \Leftrightarrow P(A \setminus B) $ \\ \\
        Lemma 1.5 (Skript):\\ $ \forall A,B \subset \Omega : P(A \setminus B) = P(A) - P(A \cap B) $ \\
        $ \Rightarrow P(A \cap B^{\mathrm{c}}) = P(A) - P(A \cap B) $ \\
        $ \Leftrightarrow P(A \cap B^{\mathrm{c}}) = P(A) - (P(A) + P(B) - P(A \cup B)) $ \\
        $ = 0.25 - (0.25 + 0.45 - 0.5) = 0.05 $ \\
        \item
        $ P(A^{\mathrm{c}} \cap B^{\mathrm{c}}) = P((A \cup B)^{\mathrm{c}}) = 1 - (A \cup B) $ \\
        $ = 1 - 0.5 = 0.5 $
        \item
        $ P((A \cap B^{\mathrm{c}}) \cup (A^{\mathrm{c}} \cap B)) = P(A^{\mathrm{c}} \cap B) + P(A^{\mathrm{c}} \cap B) $ \\
        $ = 0.05 + P(B) - (P(A) + P(B) - P(A \cup B)) = 0.05 - P(A) + P(A \cup B) $ \\
        $ = 0.5 - 0.2 = 0.3 $
    \end{itemize}

    \Aufgabe{5}


%%%%%%%%%%%%%%%%%%%%%%%%%%%%%%%%%%%%%%%%%%%%%%%%%%%%%%
%%%%%%%%%%%%%%%%%%%%%%%%%%%%%%%%%%%%%%%%%%%%%%%%%%%%%%
\end{document}